\chapter{场论四大核心概念}

\section{核心工具:倒三角算子 (\texorpdfstring{$\nabla$}{nabla})}

\begin{definition}[倒三角算子 $\nabla$]
所有场论运算的核心都是这个算子。请把它看作一个\textbf{“带求导功能的向量”}。
\begin{itemize}
    \item \textbf{符号}:$\nabla$ 
    \item \textbf{定义}:$\displaystyle \nabla = \left( \frac{\partial}{\partial x}, \ \frac{\partial}{\partial y}, \ \frac{\partial}{\partial z} \right)$
    \item \textbf{心法}:像操作向量一样操作它(点积、叉积、数乘),但它的动作是“求导”而不是简单的“数值乘法”。
\end{itemize}
\end{definition}

\section{深度解析与公式}

% 这里的标题去掉了原本的 "1.",直接显示核心概念,配合 definition 环境更突出
\begin{definition}[梯度 $\operatorname{grad} u$]
\begin{itemize}
    \item \textbf{定义}:标量场 $u(x,y,z)$ 在某点的最大变化率方向。
    \item \textbf{公式}:
    \[
        \operatorname{grad} u = \nabla u = \left( \frac{\partial u}{\partial x}, \ \frac{\partial u}{\partial y}, \ \frac{\partial u}{\partial z} \right)
    \]
    \item \textbf{性质}:
    \begin{itemize}
        \item 梯度垂直于等值面(法向量 $\vec{n} = \nabla u$)。
        \item 模长 $|\nabla u|$ 是该点最大的变化率数值。
    \end{itemize}
\end{itemize}
\end{definition}

\begin{definition}[方向导数 $\frac{\partial u}{\partial l}$]
\begin{itemize}
    \item \textbf{定义}:函数 $u$ 沿\textbf{指定方向} $\vec{l}$ 的变化率。
    \item \textbf{公式(核心考点)}:
    \[
        \frac{\partial u}{\partial l} = \nabla u \cdot \vec{e}_l = |\nabla u| \cos\theta
    \]
    \begin{itemize}
        \item $\vec{e}_l$:方向 $\vec{l}$ 的\textbf{单位向量} \textcolor{red}{(做题必坑点:记得单位化!)}。
        \item $\theta$:梯度与方向 $\vec{l}$ 的夹角。
    \end{itemize}
    \item \textbf{三者关系}:
    \begin{itemize}
        \item $\theta = 0$(同向):方向导数最大($= |\nabla u|$)。
        \item $\theta = \pi$(反向):方向导数最小($= -|\nabla u|$)。
        \item $\theta = \frac{\pi}{2}$(垂直/沿等高线):方向导数 $= 0$。
    \end{itemize}
\end{itemize}
\end{definition}

\begin{definition}[散度 $\operatorname{div} \vec{A}$]
\begin{itemize}
    \item \textbf{定义}:通量对体积的变化率,描述场的“发散”程度。
    \item \textbf{公式(点积法则)}:
    设 $\vec{A} = (P, Q, R)$,
    \[
        \operatorname{div} \vec{A} = \nabla \cdot \vec{A} = \frac{\partial P}{\partial x} + \frac{\partial Q}{\partial y} + \frac{\partial R}{\partial z}
    \]
    \item \textbf{物理意义}:$>0$ 发散(有源);$<0$ 汇聚(有汇);$=0$ 无源场。
\end{itemize}
\end{definition}

\begin{definition}[旋度 $\operatorname{rot} \vec{A}$ / $\operatorname{curl} \vec{A}$]
\begin{itemize}
    \item \textbf{定义}:环量对面密度的极限,描述场的“旋转”强度。
    \item \textbf{公式(叉积法则 - 背行列式)}:
    \[
        \operatorname{rot} \vec{A} = \nabla \times \vec{A} = \begin{vmatrix} \vec{i} & \vec{j} & \vec{k} \\ \frac{\partial}{\partial x} & \frac{\partial}{\partial y} & \frac{\partial}{\partial z} \\ P & Q & R \end{vmatrix}
    \]
    \[
        = \left(\frac{\partial R}{\partial y} - \frac{\partial Q}{\partial z}\right)\vec{i} - \left(\frac{\partial R}{\partial x} - \frac{\partial P}{\partial z}\right)\vec{j} + \left(\frac{\partial Q}{\partial x} - \frac{\partial P}{\partial y}\right)\vec{k}
    \]
\end{itemize}
\end{definition}


\section{两个重要的“恒等于零” (必考)}

% 使用 property 环境,颜色与 definition 不同,层次更丰富
\begin{property}[场论重要恒等式]
\begin{enumerate}
    \item \textbf{梯度的旋度为零}:
    \[ \operatorname{rot}(\operatorname{grad} u) = \nabla \times (\nabla u) = \vec{0} \]
    \textit{记忆}:$\vec{a} \times \vec{a} = \vec{0}$(自己不能绕着自己转)。\\
    \textit{意义}:梯度场(保守场,如重力场、静电场)是无旋的。
    
    \item \textbf{旋度的散度为零}:
    \[ \operatorname{div}(\operatorname{rot} \vec{A}) = \nabla \cdot (\nabla \times \vec{A}) = 0 \]
    \textit{记忆}:$\vec{a} \cdot (\vec{a} \times \vec{b}) = 0$(垂直向量点积为0)。\\
    \textit{意义}:旋涡场(如磁场)没有源头,磁感线是闭合的。
\end{enumerate}
\end{property}