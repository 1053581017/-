\chapter{算子法求解微分方程特解的“四大杀手锏”规则}
\begin{center}
    
\end{center}

\section{通用共振处理逻辑(求导法则)}
当直接代入导致分母 $L(D) = 0$ 时,使用“分母求导”公式。

\textbf{公式:}
\[
\frac{1}{L(D)} f(x) = x \cdot \frac{1}{L'(D)} f(x)
\]

\begin{itemize}
    \item \textbf{操作口诀:} 分母为零别慌张,\textbf{分子乘 $x$,分母求导},然后再试一次。如果还为 0,就再乘 $x$ 再求导。
\end{itemize}

\section{规则一:指数函数 $f(x) = e^{\lambda x}$ (代入规则)}

\textbf{1. 常规情况 ($L(\lambda) \neq 0$):}
直接用 $\lambda$ 代替 $D$。
\[
y^* = \frac{1}{L(\lambda)} e^{\lambda x}
\]

\textbf{2. 共振情况 ($L(\lambda) = 0$):}
使用求导法则。
\[
y^* = x \cdot \frac{1}{L'(\lambda)} e^{\lambda x}
\]
(注:若 $L'(\lambda)$ 仍为 0,则继续求导变为 $x^2 \cdot \frac{1}{L''(\lambda)} e^{\lambda x}$)

\begin{itemize}
    \item \textbf{例:} $(D-2)^2 y = e^{2x}$
    \begin{itemize}
        \item \textbf{分析:} 代入 $D=2$ 分母为 0。
        \item \textbf{一次求导:} $x \cdot \frac{1}{2(D-2)} e^{2x}$ (再次代入 $D=2$ 仍为 0)
        \item \textbf{二次求导:} $x^2 \cdot \frac{1}{2} e^{2x} = \frac{1}{2}x^2 e^{2x}$
    \end{itemize}
\end{itemize}

\section{规则二:三角函数 $f(x) = \sin(\omega x)$ 或 $\cos(\omega x)$ (平方代入规则)}

\textbf{1. 常规情况 ($L(-\omega^2) \neq 0$):}
如果算子 $L(D)$ 中只含有 $D^2$ 的项(或者能化简出 $D^2$),可以用 $-\omega^2$ 代替 $D^2$。

\textbf{2. 共振情况 ($L(-\omega^2) = 0$):}
使用求导法则(注意是对 $D$ 求导)。
\[
y^* = x \cdot \frac{1}{L'(D)} \sin(\omega x)
\]
通常求导后分母会变成含 $D$ 的一阶式,此时利用 $\frac{1}{D} = \int dx$ 进行积分。

\begin{itemize}
    \item \textbf{例:} $(D^2 + 4)y = \sin(2x)$
    \begin{itemize}
        \item \textbf{分析:} 代入 $D^2 = -4$ 分母为 0。
        \item \textbf{分母求导:} $(D^2+4)' = 2D$
        \item \textbf{应用法则:} 
        \[
        y^* = x \cdot \frac{1}{2D} \sin(2x) = \frac{x}{2} \int \sin(2x) \, dx = -\frac{x}{4}\cos(2x)
        \]
    \end{itemize}
\end{itemize}

\section{规则三:多项式 $f(x) = P_n(x)$ (级数展开规则)}

利用泰勒级数展开,将逆算子 $\frac{1}{L(D)}$ 展开成 $D$ 的幂级数。

\textbf{★ 关键:如何判断展开阶数?}
看多项式 $P_n(x)$ 的最高次数 $n$,级数展开只需\textbf{保留到 $D^n$ 项}。
\begin{itemize}
    \item \textbf{原理:} $D^{n+1} P_n(x) = 0$(更高阶导数为 0,直接舍弃)。
    \item \textbf{速判:} 
    \begin{itemize}
        \item 针对 $x$ (1次) $\rightarrow$ 保留到 $D$。
        \item 针对 $x^2$ (2次) $\rightarrow$ 保留到 $D^2$。
    \end{itemize}
\end{itemize}

\textbf{1. 常规情况(有常数项):}
直接按泰勒级数展开。

\textbf{2. 共振情况(无常数项):}
当算子最低次项为 $D^k$ 时(对应 0 特征根),使用\textbf{提公因式法}。

\begin{itemize}
    \item \textbf{步骤:}
    \begin{enumerate}
        \item 提出分母中最低次的 $D^k$。
        \item 对剩下的部分进行级数展开(保留至 $D^n$)。
        \item 最后进行 $k$ 次积分(因为 $\frac{1}{D}$ 等价于积分)。
    \end{enumerate}
    \item \textbf{例:} $(D^2 - D)y = x$
    \begin{itemize}
        \item \textbf{推导:} 提出 $D$,剩下 $\frac{1}{D-1}$ 对 $x$ (1次) 展开至 $D$ 项。
        \[
        y^* = \frac{1}{D(D-1)} x = \frac{1}{D} \left[ -(1+D) \right] x = -\frac{1}{D}(x+1) = -(\frac{x^2}{2} + x)
        \]
    \end{itemize}
\end{itemize}

\section{规则四:移位定理(Shift Theorem)}
当 $f(x)$ 是指数函数与其他函数的乘积时,可以将指数函数“移”到算子左边,同时算子中的 $D$ 变为 $D+\lambda$。

\textbf{公式:}
\[
\frac{1}{L(D)} [e^{\lambda x} v(x)] = e^{\lambda x} \frac{1}{L(D+\lambda)} v(x)
\]

\begin{itemize}
    \item \textbf{应用场景:} 处理乘积形式,或者用于化简复杂的共振。
    \item \textbf{与求导法则联动:} 如果移位后,剩下的算子对 $v(x)$ 处理时依然分母为 0,则继续对剩下的部分使用求导法则。
    \item \textbf{例:} $(D-1)y = e^x$
    \begin{itemize}
        \item \textbf{方法 A (移位法):} 看作 $e^x \cdot 1$。
        \[
        y^* = e^x \frac{1}{(D+1)-1} \cdot 1 = e^x \frac{1}{D} \cdot 1 = x e^x
        \]
        \item \textbf{方法 B (直接求导法):} 直接看 $e^x$,代入 $D=1$ 分母为 0。
        \[
        y^* = x \cdot \frac{1}{(D-1)'} e^x = x \cdot \frac{1}{1} e^x = x e^x
        \]
    \end{itemize}
\end{itemize}

\section{总结:什么时候用什么?}
\begin{itemize}
    \item 看到 $e^{\lambda x}$ $\rightarrow$ \textbf{直接代}。分母为 0? $\rightarrow$ \textbf{分子乘 $x$,分母求导}。
    \item 看到 $\sin / \cos$ $\rightarrow$ \textbf{代 $D^2$}。分母为 0? $\rightarrow$ \textbf{分子乘 $x$,分母求导}(然后积分)。
    \item 看到 $e^{\lambda x} \cdot \dots$ $\rightarrow$ \textbf{先移位},把 $e$ 移走,再处理剩下的。
    \item 看到多项式 $\rightarrow$ \textbf{展开}(保留至 $D^n$)。没常数项? $\rightarrow$ \textbf{提 $D$ 积分}。
\end{itemize}