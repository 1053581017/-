\chapter{场论四大核心概念}

\section{核心工具:倒三角算子 ($\nabla$)}

所有场论运算的核心都是这个算子。请把它看作一个\textbf{“带求导功能的向量”}。

\begin{itemize}
    \item \textbf{符号}:$\nabla$ 
    \item \textbf{定义}:$\nabla = \left( \frac{\partial}{\partial x}, \ \frac{\partial}{\partial y}, \ \frac{\partial}{\partial z} \right)$
    \item \textbf{心法}:像操作向量一样操作它(点积、叉积、数乘),但它的动作是“求导”而不是简单的“数值乘法”。
\end{itemize}

\section{深度解析与公式}

\subsection{1. 梯度 ($\text{grad } u$)}
\begin{itemize}
    \item \textbf{定义}:标量场 $u(x,y,z)$ 在某点的最大变化率方向。
    \item \textbf{公式}:
    \[
        \text{grad } u = \nabla u = \left( \frac{\partial u}{\partial x}, \ \frac{\partial u}{\partial y}, \ \frac{\partial u}{\partial z} \right)
    \]
    \item \textbf{性质}:
    \begin{itemize}
        \item 梯度垂直于等值面(法向量 $\vect{n} = \nabla u$)。
        \item 模长 $|\nabla u|$ 是该点最大的变化率数值。
    \end{itemize}
\end{itemize}

\subsection{2. 方向导数 ($\frac{\partial u}{\partial l}$)}
\begin{itemize}
    \item \textbf{定义}:函数 $u$ 沿\textbf{指定方向} $\vect{l}$ 的变化率。
    \item \textbf{公式(核心考点)}:
    \[
        \frac{\partial u}{\partial l} = \nabla u \cdot \vect{e}_l = |\nabla u| \cos\theta
    \]
    \begin{itemize}
        \item $\vect{e}_l$:方向 $\vect{l}$ 的\textbf{单位向量} \textcolor{red}{(做题必坑点:记得单位化!)}。
        \item $\theta$:梯度与方向 $\vect{l}$ 的夹角。
    \end{itemize}
    \item \textbf{三者关系}:
    \begin{itemize}
        \item $\theta = 0$(同向):方向导数最大($= |\nabla u|$)。
        \item $\theta = \pi$(反向):方向导数最小($= -|\nabla u|$)。
        \item $\theta = \frac{\pi}{2}$(垂直/沿等高线):方向导数 $= 0$。
    \end{itemize}
\end{itemize}

\subsection{3. 散度 ($\text{div } \vect{A}$)}
\begin{itemize}
    \item \textbf{定义}:通量对体积的变化率,描述场的“发散”程度。
    \item \textbf{公式(点积法则)}:
    设 $\vect{A} = (P, Q, R)$,
    \[
        \text{div } \vect{A} = \nabla \cdot \vect{A} = \frac{\partial P}{\partial x} + \frac{\partial Q}{\partial y} + \frac{\partial R}{\partial z}
    \]
    \item \textbf{物理意义}:$>0$ 发散(有源);$<0$ 汇聚(有汇);$=0$ 无源场。
\end{itemize}

\subsection{4. 旋度 ($\text{rot } \vect{A}$ / $\text{curl } \vect{A}$)}
\begin{itemize}
    \item \textbf{定义}:环量对面密度的极限,描述场的“旋转”强度。
    \item \textbf{公式(叉积法则 - 背行列式)}:
    \[
        \text{rot } \vect{A} = \nabla \times \vect{A} = \begin{vmatrix} \vect{i} & \vect{j} & \vect{k} \\ \frac{\partial}{\partial x} & \frac{\partial}{\partial y} & \frac{\partial}{\partial z} \\ P & Q & R \end{vmatrix}
    \]
    \[
        = \left(\frac{\partial R}{\partial y} - \frac{\partial Q}{\partial z}\right)\vect{i} - \left(\frac{\partial R}{\partial x} - \frac{\partial P}{\partial z}\right)\vect{j} + \left(\frac{\partial Q}{\partial x} - \frac{\partial P}{\partial y}\right)\vect{k}
    \]
\end{itemize}


\section{两个重要的“恒等于零” (必考)}

\begin{enumerate}
    \item \textbf{梯度的旋度为零}:
    \[ \text{rot}(\text{grad } u) = \nabla \times (\nabla u) = \vect{0} \]
    \textit{记忆}:$\vect{a} \times \vect{a} = \vect{0}$(自己不能绕着自己转)。\\
    \textit{意义}:梯度场(保守场,如重力场、静电场)是无旋的。
    
    \item \textbf{旋度的散度为零}:
    \[ \text{div}(\text{rot } \vect{A}) = \nabla \cdot (\nabla \times \vect{A}) = 0 \]
    \textit{记忆}:$\vect{a} \cdot (\vect{a} \times \vect{b}) = 0$(垂直向量点积为0)。\\
    \textit{意义}:旋涡场(如磁场)没有源头,磁感线是闭合的。
\end{enumerate}
