\chapter{标注小技巧}\label{ch:crossref}
\section*{第一部分:曲线积分 (Curve Integrals)}

\subsection*{一、第一类曲线积分(对弧长的曲线积分)}
% 使用 tabularx 环境,使表格宽度适应文本宽度
% 列定义:p{固定宽度} 和 X{自适应宽度}
\begin{table}[htbp]
    \centering
    \begin{tabularx}{\textwidth}{p{2.5cm} p{3.5cm} X p{4.5cm}}
    \toprule
    \textbf{一级分类} & \textbf{具体题型/场景} & \textbf{核心解法/步骤} & \textbf{注意/技巧} \\
    \midrule
    
    对称性与两点法 & 积分曲线L关于坐标轴(如x轴、y轴)对称。 & 根据被积函数 $f(x,y)$ 关于对应变量的奇偶性判断积分结果。
    \begin{enumerate}
        \item 若为奇函数,积分为0。
        \item 若为偶函数,积分等于在单侧曲线上积分的2倍。
    \end{enumerate} & \textbf{【两点法】}(简化计算/推导结论)
    \newline 用于快速判断对称性,无需记忆公式:
    \begin{enumerate}
        \item 写出点P关于对称轴的对称点P'。
        \item 写出P和P'对应的被积分式(\textbf{注意:第一类积分中ds不变})。
        \item 两式相加。若和为零,则总积分为0。
    \end{enumerate} \\
    \midrule
    
    对称性与两点法 & \textbf{特殊对称性(变量对称性)}:积分曲线L关于直线 $y=x$ 对称。 & 利用变量轮换性质:
    \newline $\int_{L}f(x,y)ds=\int_{L}f(y,x)ds$。 & \textbf{【注】}:若曲线方程 $F(x,y)=0$ 满足 $F(x,y)=F(y,x)$ 或 $F(x,y)=-F(y,x)$,则曲线关于 $y=x$ 对称。 \\
    \midrule
    
    计算方法 & L由显函数 $y=g(x)$, $a\le x\le b$ 给出。 & 化为定积分:
    \newline $\int_{L}f(x,y)ds=\int_{a}^{b}f(x,g(x))\sqrt{1+[g^{\prime}(x)]^{2}}dx$。 & 核心是计算弧微分 $ds$。 \\
    \midrule
    
    计算方法 & L由参数方程 $\begin{cases}x=x(t)\\ y=y(t)\end{cases}$, $\alpha\le t\le\beta$ 给出。 & 化为定积分:
    \newline $\int_{L}f(x,y)ds=\int_{\alpha}^{\beta}f(x(t),y(t))\sqrt{[x^{\prime}(t)]^{2}+[y^{\prime}(t)]^{2}}dt$。 & 适用于复杂的平面曲线或空间曲线。 \\
    \midrule
    
    计算方法 & L由极坐标 $r=r(\theta)$, $\alpha\le\theta\le\beta$ 给出。 & 化为定积分:
    \newline $\int_{L}f(x,y)ds=\int_{\alpha}^{\beta}f(r\cos\theta,r\sin\theta)\sqrt{r^{2}+r^{\prime2}}d\theta$。 & 适用于积分路径为圆或与圆相关的曲线。 \\
    \midrule
    
    简化运算技巧 & 被积表达式复杂。 & \textbf{代入曲线方程简化运算}:将被积函数中的部分表达式用曲线方程进行替换或化简(例如利用 $x^2+y^2=R^2$)。 & 对称性与代入方程简化通常结合运用。 \\
    
    \bottomrule
    \end{tabularx}
\end{table}

\subsection*{二、第二类曲线积分(对坐标的曲线积分)}
\begin{table}[htbp]
    \centering
    \begin{tabularx}{\textwidth}{p{2.5cm} p{3.5cm} X p{4.5cm}}
    \toprule
    \textbf{一级分类} & \textbf{具体题型/场景} & \textbf{核心解法/步骤} & \textbf{注意/技巧} \\
    \midrule
    
    对称性与两点法 & 有向积分曲线L关于坐标轴对称。 & 根据被积函数关于对应变量的奇偶性以及积分方向判断积分结果。 & \textbf{【两点法】}(简化计算/推导结论)
    \begin{enumerate}
        \item 写出点P关于对称轴的对称点P'。
        \item 写出P和P'对应的被积分式。\textbf{关键注意}:由于曲线有向,对称点的微分项可能会变号(例如关于x轴对称时,若路径方向相反,dx可能变为-dx)。
        \item 两式相加。若和为零,则总积分为0。
    \end{enumerate} \\
    \midrule
    
    计算方法 & L由显函数 $y=g(x)$ 给出,起点对应x=a,终点对应x=b。 & 化为定积分:
    \newline $\int_{L}Pdx+Qdy=\int_{a}^{b}[P(x,g(x))+Q(x,g(x))g^{\prime}(x)]dx$。 & 注意积分上下限必须与曲线方向一致。 \\
    \midrule
    
    计算方法 & L由参数方程给出,起点对应$t=\alpha$,终点对应$t=\beta$。 & 化为定积分:
    \newline $\int_{L}Pdx+Qdy=\int_{\alpha}^{\beta}[P(x(t),y(t))x^{\prime}(t)+Q(x(t),y(t))y^{\prime}(t)]dt$。 & \textbf{【曲线方程参数化】}:常用于处理空间曲线积分(如两个曲面的交线,例如柱面与平面相交)。需要根据题意正确设定参数方程和方向。 \\
    \midrule
    
    计算方法 & L由极坐标 $r=r(\theta)$ 给出,起点对应$\theta=\alpha$,终点对应$\theta=\beta$。 & 化为定积分(根据讲义公式):
    \newline $\int_{L}Pdx+Qdy=\int_{\alpha}^{\beta}\{P[r(\theta)\cos\theta]^{\prime}+Q[r(\theta)\sin\theta]^{\prime}\}d\theta$。 & 注意将x,y代入后对 $\theta$ 求导。 \\
    
    \bottomrule
    \end{tabularx}
\end{table}

\subsection*{三、格林公式及其应用技巧}
\begin{table}[htbp]
    \centering
    \begin{tabularx}{\textwidth}{p{3cm} p{3cm} X p{4.5cm}}
    \toprule
    \textbf{一级分类} & \textbf{具体题型/场景} & \textbf{核心解法/步骤} & \textbf{注意/技巧} \\
    \midrule
    
    格林公式 & 平面闭区域D上的第二类曲线积分。P, Q在D上具有一阶连续偏导数。 & 将沿正向边界L的曲线积分转化为二重积分:
    \newline $\oint_{L}Pdx+Qdy=\iint_{D}(\frac{\partial Q}{\partial x}-\frac{\partial P}{\partial y})dxdy$。 & L必须是D的\textbf{所有}正向(通常是逆时针)边界曲线。 \\
    \midrule
    
    格林公式应用技巧一:补线 & 曲线L不是闭曲线,且直接计算不易。 & 
    \begin{enumerate}
        \item 作辅助路径 $\Gamma$,使L与$\Gamma$构成闭曲线,围成区域D。
        \item 利用格林公式计算闭曲线积分。
        \item 原积分 = 格林公式结果 - 辅助路径上的积分。
    \end{enumerate} & 
    \begin{enumerate}
        \item 要求D内无奇点。
        \item \textbf{【$\Gamma$的选取】}:使得在 $\Gamma$ 上的积分容易计算,一般取折线或线段。注意构造时保持正向(逆时针)。
    \end{enumerate} \\
    \midrule
    
    格林公式应用技巧二:补圈挖点 & 闭曲线L所围区域内有奇点(P, Q偏导数不连续或无定义)。 & 
    \begin{enumerate}
        \item 在L内部,补上一条包含奇点的小闭曲线 $\Gamma$(挖去奇点),使L和$\Gamma$之间的区域D不含奇点。
        \item 在D上应用格林公式。
        \item $\int_{L}Pdx+Qdy=\int_{\Gamma}Pdx+Qdy+\iint_{D}(\frac{\partial Q}{\partial x}-\frac{\partial P}{\partial y})dxdy$。
    \end{enumerate} & \textbf{【$\Gamma$的选取】}:使得在 $\Gamma$ 上的积分容易计算。通常选取能消去分母的曲线。例如被积式为 $\frac{xdy-ydx}{4x^{2}+y^{2}}$,可取 $\Gamma$ 为 $4x^{2}+y^{2}=1$。注意L和-$\Gamma$是D的正向边界。 \\
    
    \bottomrule
    \end{tabularx}
\end{table}

\subsection*{四、路径无关问题与斯托克斯公式}
\begin{table}[htbp]
    \centering
    \begin{tabularx}{\textwidth}{p{2.5cm} p{3.5cm} X p{4.5cm}}
    \toprule
    \textbf{一级分类} & \textbf{具体题型/场景} & \textbf{核心解法/步骤} & \textbf{注意/技巧} \\
    \midrule
    
    路径无关问题 & 判断积分与路径无关,或利用该性质求积分、求参数/函数。 & 在单连通域G内,若P, Q有连续偏导数,则路径无关等价于以下条件(满足其一即可):
    \begin{enumerate}
        \item 沿G内任意闭曲线积分为0。
        \item $Pdx+Qdy$ 是某函数 $U(x,y)$ 的全微分 $dU(x,y)$。
        \item $\frac{\partial P}{\partial y}=\frac{\partial Q}{\partial x}$ 在G内处处成立。
    \end{enumerate} & 若积分与路径无关,可以直接将积分写为 $\int_{(x_{1},y_{1})}^{(x_{2},y_{2})}$,并可选择最简单的路径(如折线)进行计算,或求出原函数U。 \\
    \midrule
    
    斯托克斯公式 & 空间有向闭曲线 $\Gamma$ 上的第二类曲线积分。 & 将曲线积分转化为以 $\Gamma$ 为边界的有向曲面 $\Sigma$ 上的曲面积分(计算旋度):
    \newline $\oint_{\Gamma}Pdx+Qdy+Rdz=\iint_{\Sigma}\begin{vmatrix}dydz&dxdz&dxdy\\ \frac{\partial}{\partial x}&\frac{\partial}{\partial y}&\frac{\partial}{\partial z}\\ P&Q&R\end{vmatrix}$ & 
    \begin{enumerate}
        \item $\Gamma$ 的正向与 $\Sigma$ 的侧符合右手规则。
        \item P, Q, R需在 $\Sigma$ 上具有一阶连续偏导数。
    \end{enumerate} \\
    \midrule
    
    斯托克斯公式应用:补线 & 空间曲线 $\Gamma$ 不是闭曲线。 & 补线使之成为闭曲线,然后应用斯托克斯公式,再减去补线上积分。 & 类似于格林公式的补线技巧,但应用于空间曲线。 \\
    
    \bottomrule
    \end{tabularx}
\end{table}


\newpage
\section*{第二部分:曲面积分 (Surface Integrals)}

\subsection*{一、第一类曲面积分(对面积的曲面积分)}
\begin{table}[htbp]
    \centering
    \begin{tabularx}{\textwidth}{p{2.5cm} p{3.5cm} X p{4.5cm}}
    \toprule
    \textbf{一级分类} & \textbf{具体题型/场景} & \textbf{核心解法/步骤} & \textbf{注意/技巧} \\
    \midrule
    
    对称性与两点法 & 积分曲面 $\Sigma$ 关于坐标平面(如xOy面)对称。 & 根据被积函数关于对应变量(如z)的奇偶性判断积分结果(积分为0或2倍)。 & \textbf{【两点法】}(判断第一类曲面积分对称性)
    \begin{enumerate}
        \item 写出点P关于对称面的对称点P'。
        \item 写出P和P'对应的被积分式(\textbf{注意:dS不变})。
        \item 两式相加。若和为零,则总积分为0。
    \end{enumerate} \\
    \midrule
    
    对称性与两点法 & \textbf{特殊对称性(变量对称性)}:积分曲面 $\Sigma$ 关于特定平面对称(如 $y=x$)。 & 利用变量轮换性质:
    \newline $\iint_{\Sigma}f(x,y,z)dS=\iint_{\Sigma}f(y,x,z)dS$。 & \textbf{【注】}:若曲面方程 $F(x,y,z)=0$ 满足 $F(x,y,z)=F(y,x,z)$,则曲面关于 $y=x$ 对称。 \\
    \midrule
    
    计算方法:投影法 & 积分曲面由 $z=z(x,y)$ 给出,投影为 $D_{xy}$。 & \textbf{【投影法】}(将曲面积分化为二重积分):
    \newline $\iint_{\Sigma}f(x,y,z)dS=\iint_{D_{xy}}f(x,y,z(x,y))\sqrt{1+z_{x}^{2}+{z_{y}}^{2}}dxdy$。 & 核心是计算面积微元 $dS$。同样可以向yOz面或xOz面投影。 \\
    \midrule
    
    特殊应用 & 计算特殊曲面(如柱面被截)的面积。 & \textbf{【弧微分求曲面面积】}:利用弧微分的思想计算曲面面积。 & 适用于计算柱面等侧面积。 \\
    
    \bottomrule
    \end{tabularx}
\end{table}

\subsection*{二、第二类曲面积分(对坐标的曲面积分)}
\begin{table}[htbp]
    \centering
    \begin{tabularx}{\textwidth}{p{3cm} p{3cm} X p{4.5cm}}
    \toprule
    \textbf{一级分类} & \textbf{具体题型/场景} & \textbf{核心解法/步骤} & \textbf{注意/技巧} \\
    \midrule
    
    对称性与两点法 & 积分曲面 $\Sigma$(有向)关于坐标平面对称。 & 根据被积函数的奇偶性以及曲面方向判断积分结果。 & \textbf{【两点法】}(判断第二类曲面积分对称性)
    \begin{enumerate}
        \item 写出点P关于对称面的对称点P'。
        \item 写出P和P'对应的被积分式。\textbf{关键注意}:由于曲面有向,对称点的微分项可能会变号(例如关于xOy面对称时,上侧的dxdy在下侧可能对应-dxdy)。
        \item 两式相加。若和为零,则总积分为0。
    \end{enumerate} \\
    \midrule
    
    计算方法一:投影法 & 积分曲面由显函数给出(如 $z=z(x,y)$)。 & \textbf{【投影法】}(分别计算三个分量):
    \newline 分别计算 $\iint_{\Sigma}P dydz, \iint_{\Sigma}Q dxdz, \iint_{\Sigma}R dxdy$。
    \newline 以R为例:$\iint_{\Sigma}Rdxdy=\pm\iint_{D_{xy}}R(x,y,z(x,y))dxdy$。 & \textbf{【符号判断】}:
    \newline 当曲面法向量与对应轴(如z轴)正向夹角为锐角时取“+”,钝角时取“-”。 \\
    \midrule
    
    计算方法二:利用两类曲面积分的联系 & 通用计算方法,特别是曲面法向量容易求得时。 & 将第二类曲面积分转化为第一类曲面积分:
    \newline $\iint_{\Sigma}Pdydz+Qdxdz+Rdxdy=\iint_{\Sigma}(P\cos\alpha+Q\cos\beta+R\cos\gamma)dS$。
    \newline 其中 $\mathbf{n}=(\cos\alpha,\cos\beta,\cos\gamma)$ 是曲面指定侧的单位法向量。 & 
    \begin{enumerate}
        \item 关键是计算单位法向量 $\mathbf{n}$(讲义提供了显函数和隐函数的计算公式)。
        \item \textbf{【正负号判断】}:计算出的法向量需要通过图像判断其与指定侧的方向是相同(取正号)还是相反(取负号)。
    \end{enumerate} \\
    \midrule
    
    计算方法三:合一投影法 & 积分曲面由显函数(如 $z=z(x,y)$)或隐函数给出。 & \textbf{【合一投影法】}(将三个分量同时投影到一个坐标面):
    \newline 若投影到$D_{xy}$(设 $\Sigma: z=z(x,y)$):
    \newline $\iint_{\Sigma}Pdydz+Qdxdz+Rdxdy=\iint_{D_{xy}}[P\cdot(-z_{x})+Q\cdot(-z_{y})+R\cdot1]dxdy$。 & 这个方法用起来和方法一差不多。对于隐函数 $F(x,y,z)=0$,可以用偏导数的比值代换。 \\
    
    \bottomrule
    \end{tabularx}
\end{table}

\subsection*{三、高斯公式及其应用技巧}
\begin{table}[htbp]
    \centering
    \begin{tabularx}{\textwidth}{p{3cm} p{3cm} X p{4.5cm}}
    \toprule
    \textbf{一级分类} & \textbf{具体题型/场景} & \textbf{核心解法/步骤} & \textbf{注意/技巧} \\
    \midrule
    
    计算方法四:高斯公式 & 空间闭区域 $\Omega$ 上的曲面积分。P, Q, R在 $\Omega$ 上具有一阶连续偏导数。 & \textbf{【高斯公式】}(将闭曲面积分转化为三重积分):
    \newline $\iint_{\Sigma}Pdydz+Qdzdx+Rdxdy = \iiint_{\Omega}(\frac{\partial P}{\partial x}+\frac{\partial Q}{\partial y}+\frac{\partial R}{\partial z})dv$。 & $\Sigma$ 必须是 $\Omega$ 的所有外侧边界曲面。 \\
    \midrule
    
    高斯公式应用技巧一:补面 & 曲面 $\Sigma$ 不是闭曲面,且直接计算不易。 & 
    \begin{enumerate}
        \item 作辅助曲面 $\Sigma'$,使 $\Sigma+\Sigma'$ 构成闭曲面,围成区域 $\Omega$。
        \item 利用高斯公式计算闭曲面积分。
        \item 原积分 = 高斯公式结果 - 辅助曲面 $\Sigma'$ 上的积分。
    \end{enumerate} & 
    \begin{enumerate}
        \item $\Sigma$ 与 $\Sigma'$ 的方向应构成 $\Omega$ 的外侧。要求$\Omega$内无奇点。
        \item \textbf{【$\Sigma'$的选取】}:使得在 $\Sigma'$ 上的积分容易计算,一般取平面。
    \end{enumerate} \\
    \midrule
    
    高斯公式应用技巧二:挖点 & 闭曲面 $\Sigma$ 所围区域内有奇点。 & 
    \begin{enumerate}
        \item 在 $\Sigma$ 内部,补上一个包含奇点的小闭曲面 $\Sigma'$(挖去奇点),使 $\Sigma$ 和 $\Sigma'$ 之间的区域 $\Omega$ 不含奇点。
        \item 在 $\Omega$ 上应用高斯公式。
        \item 原积分 = $\Sigma'$ 上的积分 + 高斯公式结果。
    \end{enumerate} & \textbf{【$\Sigma'$的选取】}:使得在 $\Sigma'$ 上的积分容易计算。通常选取能消去分母的曲面。例如被积式分母为 $(x^{2}+y^{2}+z^{2})^{3/2}$,可取 $\Sigma'$ 为小球面 $x^{2}+y^{2}+z^{2}=\epsilon^{2}$。注意方向的配置。 \\
    
    \bottomrule
    \end{tabularx}
\end{table}