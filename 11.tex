% --- 你的内容 ---
\chapter{无穷级数}
\section{常数项级数}
\subsection{常数项方程}

级数$\sum_{n=1}^{\infty} u_n$ 的前 $n$ 项和
\begin{equation}\tag{11.1}
S_n=\sum_{k=1}^{n}u_k
  =u_1+u_2+\cdots+u_n,\qquad (n=1,2,\ldots)
\end{equation}
称为无穷级数的部分和。

若数项级数 $\sum_{n=1}^{\infty} u_n$ 的部分和数列 $\{S_n\}$ 的极限
$\lim_{n\to\infty}S_n$ 存在,则称级数 $\sum_{n=1}^{\infty} u_n$ 收敛;
否则称级数 $\sum_{n=1}^{\infty} u_n$ 发散。当级数 $\sum_{n=1}^{\infty} u_n$ 收敛时,称极限值 $\lim_{n\to\infty}S_n$ 为此级数的和,
即
\[
\sum_{n=1}^{\infty} u_n
  = \lim_{n\to\infty} S_n
  = \lim_{n\to\infty}\sum_{k=1}^{n} u_k
  = S .
\]

% --- 我帮你添加的“压缩映射”部分 ---

\section{压缩映射原理做题步骤}

\subsection{核心思想}
对于形如 $x_{n+1}=f(x_{n})$ 的数列,如果能证明函数 $f(x)$ 在某个区间上是一个“压缩映射”,即 $|f'(x)| \le k < 1$,那么数列 $\{x_n\}$ 必定会收敛到 $f(x)$ 的不动点 $a$(即 $f(a)=a$)。

这种方法的本质是“先斩后奏”:先假设极限 $a$ 存在并求出它,然后再证明数列确实收敛到 $a$。

\subsection{解题步骤}
\begin{enumerate}
    % --- 第 1 步 ---
    \item \textbf{【先斩后奏,求出不动点 $a$】}
    \begin{itemize}
        \item 假设 $\lim_{n \to \infty} x_n = a$ 存在。
        \item 在递推公式 $x_{n+1} = f(x_n)$ 两边同时取极限,得到 $a = f(a)$。
        \item 解此方程,求出不动点 $a$ 的值。(注意:有时需要根据 $x_n$ 的有界性,从多个解中筛选出真正的极限值)。
    \end{itemize}

    % --- 第 2 步 ---
    \item \textbf{【构造压缩形式】}
    \begin{itemize}
        \item 考察 $x_{n+1}$ 与 $a$ 之间的距离,即考察绝对值 $|x_{n+1} - a|$。
        \item 将 $x_{n+1} = f(x_n)$ 和 $a = f(a)$ 代入,得到:
        $$ |x_{n+1} - a| = |f(x_n) - f(a)| $$
    \end{itemize}
    
    % --- 第 3 步 ---
    \item \textbf{【应用拉格朗日中值定理 (MVT)】}
    \begin{itemize}
        \item 对 $|f(x_n) - f(a)|$ 应用拉格朗日中值定理,得到:
        $$ f(x_n) - f(a) = f'(\xi_n)(x_n - a) $$
        其中 $\xi_n$ 位于 $x_n$ 和 $a$ 之间。
        \item 将此代回第2步的等式,得到:
        $$ |x_{n+1} - a| = |f'(\xi_n)| \cdot |x_n - a| $$
    \end{itemize}
    
    % --- 第 4 步 ---
    \item \textbf{【确定压缩常数 $k$】(关键步骤)}
    \begin{itemize}
        \item 证明存在一个 \textbf{常数 $k$},满足 $0 < k < 1$,使得对于所有的 $n$,都有 $|f'(\xi_n)| \le k$。
        \item \textbf{常用技巧:} 利用题目条件(如“$f(x)$ 有连续导数”)和闭区间上连续函数的最值定理,找到 $|f'(x)|$ 在相关区间上的最大值,并证明该最大值 $k$ 严格小于 1。
    \end{itemize}
    
    % --- 第 5 步 ---
    \item \textbf{【递推与夹逼准则】}
    \begin{itemize}
        \item 得到核心不等式:
        $$ |x_{n+1} - a| \le k |x_n - a| $$
        \item 反复应用此不等式:
        $$ |x_{n+1} - a| \le k |x_n - a| \le k^2 |x_{n-1} - a| \le \dots \le k^n |x_1 - a| $$
        \item 因为 $0 < k < 1$,所以 $\lim_{n \to \infty} k^n = 0$。
        \item 根据夹逼准则(Squeeze Theorem):
        $$ 0 \le \lim_{n \to \infty} |x_{n+1} - a| \le \lim_{n \to \infty} \left( k^n |x_1 - a| \right) = 0 $$
        \item 因此,$\lim_{n \to \infty} |x_{n+1} - a| = 0$,这等价于 $\lim_{n \to \infty} x_n = a$。
    \end{itemize}
\end{enumerate}
