\chapter{Γ函数}


\section*{一、定义}

$\Gamma$ 函数(Gamma Function)是阶乘在实数域上的推广。其定义为定义在 $(0, +\infty)$ 上的广义积分:

\begin{equation}
    \Gamma(s) = \int_0^{+\infty} x^{s-1} e^{-x} dx \quad (s > 0)
\end{equation}

\noindent \textbf{注:} 积分变量为 $x$,形式必须为 $x$ 的幂次乘以 $e^{-x}$。

\section*{二、核心性质}

以下性质是解题(尤其是广义积分计算)的关键,需熟练记忆。

\subsection*{1. 递推公式}
这是 $\Gamma$ 函数最基本的性质,用于降阶:
\[
    \Gamma(s+1) = s\Gamma(s)
\]

\subsection*{2. 阶乘关系}
当 $n$ 为正整数时,$\Gamma$ 函数与阶乘有如下关系:
\[
    \Gamma(n+1) = n!
\]
\[
    \Gamma(n) = (n-1)!
\]

\subsection*{3. 重要特殊值}
\begin{itemize}
    \item \textbf{整数点:}
    \[
        \Gamma(1) = 0! = 1
    \]
    \item \textbf{半整数点(高斯积分核心):}
    \[
        \Gamma\left(\frac{1}{2}\right) = \sqrt{\pi}
    \]
\end{itemize}

\subsection*{4. 半整数推导(由性质 1 与 3 推导)}
利用递推公式 $\Gamma(s+1) = s\Gamma(s)$:
\[
    \Gamma\left(\frac{3}{2}\right) = \frac{1}{2}\Gamma\left(\frac{1}{2}\right) = \frac{\sqrt{\pi}}{2}
\]
\[
    \Gamma\left(\frac{5}{2}\right) = \frac{3}{2} \cdot \frac{1}{2}\Gamma\left(\frac{1}{2}\right) = \frac{3\sqrt{\pi}}{4}
\]

\section*{三、常用积分公式结论}

基于 $\Gamma$ 函数定义的推广公式(解题神器):

\[
    \int_0^{+\infty} x^n e^{-kx} dx = \frac{\Gamma(n+1)}{k^{n+1}} = \frac{n!}{k^{n+1}} \quad (n \in \mathbb{N}, k > 0)
\]
