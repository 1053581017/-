行列式:一、 递推公式法 (主要针对三对角行列式)
1. 适用结构:通常是三对角矩阵(即只有主对角线以及主对角线上方和下方的一条对角线有非零元素,其余全为0)。例如:$$\begin{vmatrix}
a & b & 0 & \cdots & 0 \\
c & a & b & \cdots & 0 \\
0 & c & a & \cdots & 0 \\
\vdots & \vdots & \vdots & \ddots & \vdots \\
0 & 0 & 0 & \cdots & a
\end{vmatrix}$$2. 核心步骤:建立递推关系: 按照第一行(或第一列)展开。由于第一行只有两个非零元素,展开后会得到一个 $n-1$ 阶行列式和一个 $n-2$ 阶行列式。通常形式为:$D_n = a D_{n-1} - bc D_{n-2}$。求解递推数列: 得到关系式后,将其视为数列问题求解。3. 求解技巧(图片中提到的方法):方法使用数列技巧


二、 爪形行列式求法 (及镶边行列式)这在图片 [例 1.18]、[例 1.19] 和 [例 1.21] 中出现。1. 适用结构:像“爪子”一样的形状。通常是第一行、第一列以及主对角线有非零元素,其余位置全为0(或者最后一行、最后一列、斜对角线)。$$\begin{vmatrix}
a_1 & a_2 & a_3 & \cdots & a_n \\
b_1 & c_2 & 0 & \cdots & 0 \\
b_2 & 0 & c_3 & \cdots & 0 \\
\vdots & \vdots & \vdots & \ddots & \vdots \\
b_n & 0 & 0 & \cdots & c_n
\end{vmatrix}$$2. 核心解法:直接展开法(图片中的解法):按第一行(或爪子的“柄”所在的行/列)展开。图片 [例 1.18] 展示了展开后会得到 $D_n = a_1 \prod c_i - \sum (\text{各项})$ 的形式。图片总结的公式是:$D_n = (\prod_{i=1}^n c_i) (a_0 - \sum_{i=1}^n \frac{b_i a_i}{c_i})$ (针对特定的 $a_0$ 在左上角的情况)。化为三角行列式(消元法):利用对角线上的元素 $c_i$,将第一列(或第一行)的 $b_i$ 消成 0。这通常是最通用的方法,将“爪子”剪掉,变成上三角或下三角行列式。



线性相关与线性无关:r代表长成的空间维数,列数m是向量的数量,r的维数=m的个数则线性无关,没有多余向量。r<m则有多余的向量线性相关
齐次方程的解也即向量在线性无关时的系数,存在不为零的系数则有非零解,此时线性相关r<m,只存在零解的时候则线性无关r=m


矩阵等价 (Matrix Equivalence):定义:矩阵 $A$ 经过有限次初等变换(行或列都可以)变成矩阵 $B$。充要条件:$r(A) = r(B)$(秩相等)。口诀:只要秩一样,矩阵就等价。向量组等价 (Vector Group Equivalence):定义:向量组 (I) 和向量组 (II) 可以相互线性表示(生成的空间相同)。充要条件:$r(I) = r(II) = r(I, II)$。直观理解:它们“长”在同一个空间里,且张成的范围(维数)一样。

初等行变换的核心性质:保持行空间不变: 行向量组是等价的。保持列向量的线性关系不变: $Ax=0$ 和 $Bx=0$ 同解。破坏列空间: 除非 $P$ 是单位阵,否则列空间通常会变。

公式 $(A^*)^* = |A|^{n-2} A$ 是线性代数中的一个经典证明
利用伴随矩阵的定义(推荐,适用于 $|A| \neq 0$)核心工具:伴随矩阵的基本性质:$M M^* = M^* M = |M| E$(对任意 $n$ 阶矩阵 $M$ 成立)。伴随矩阵的行列式性质:$|A^*| = |A|^{n-1}$。推导步骤:将公式应用于 $A^*$把 $A^*$ 看作一个新的矩阵 $B$,根据性质 $B B^* = |B| E$,我们可以写出:$$A^* (A^*)^* = |A^*| E$$代入 $|A^*|$ 的公式我们知道 $|A^*| = |A|^{n-1}$,将其代入上式:$$A^* (A^*)^* = |A|^{n-1} E$$两边左乘 $A$为了利用 $A A^* = |A| E$ 消除左边的 $A^*$,我们在等式两边同时左乘矩阵 $A$:$$A \left[ A^* (A^*)^* \right] = A \left( |A|^{n-1} E \right)$$结合律与化简利用矩阵乘法结合律 $(A A^*) (A^*)^* = |A|^{n-1} A$。因为 $A A^* = |A| E$,代入得:$$(|A| E) (A^*)^* = |A|^{n-1} A$$即:$$|A| (A^*)^* = |A|^{n-1} A$$得出结论当 $|A| \neq 0$ 时,我们可以从等式两边约去一个数 $|A|$:$$(A^*)^* = \frac{|A|^{n-1}}{|A|} A = |A|^{n-2} A$$

$Ax=0$ 显然能推出 $A^TAx=0$(两边左乘 $A^T$ 即可)
这个性质直接导致了两个非常重要的考研结论:同解性质:方程组 $Ax=0$ 和 $A^TAx=0$ 的解完全一样(同解)。秩相等性质(Rank):因为解空间一样大,根据“秩-零度定理”,矩阵的秩也必须一样。$$r(A) = r(A^TA) = r(A^T)$$

如果有两个矩阵,它们的特征值、行列式、迹都一样,但你怀疑它们不相似,就用$$r(A-kE) = r(B-kE)$$检查:看它们对应特征值的秩是否相等。如果不等,直接判死刑(不相似)。