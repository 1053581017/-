\documentclass[cn,10pt,citestyle=gb7714-2015,bibstyle=gb7714-2015]{0-figure/elegantbook}

\usepackage{CJKfntef}
\usepackage{amssymb}
\usepackage{array}
\usepackage[utf8]{inputenc}
%\usepackage{caption}
\usepackage{chemformula}
\usepackage{svg}
\usepackage{extarrows}
%\usepackage{0-figure/extrarelation}
\usepackage{tikz-3dplot}
\usepackage{amsmath} % 数学公式宏包
\usepackage{booktabs} % 专业表格样式
\usepackage{tabularx} % 自适应宽度表格
\usepackage{geometry} % 设置页面几何
\usepackage{enumitem} % 控制列表间距

% 设置页边距
\geometry{a4paper, margin=1in}

% 设置在表格单元格中使用列表时的间距
\setlist[itemize]{leftmargin=*, itemsep=0pt, topsep=2pt}
\setlist[enumerate]{leftmargin=*, itemsep=0pt, topsep=2pt}
\usetikzlibrary{cd}
\cover{0-figure/cover.jpg}
\logo{0-figure/science-and-mind.jpg}
\input{newcommand}

\title{LaTeX+Overleaf考研数学一}
\author{小小叉车}
\date{\today}

\begin{document}
\maketitle
\frontmatter

\newpage

\tableofcontents
\mainmatter


\input{1}
\newpage

\input{2}
\newpage

\chapter{Γ函数}


\section*{一、定义}

$\Gamma$ 函数(Gamma Function)是阶乘在实数域上的推广。其定义为定义在 $(0, +\infty)$ 上的广义积分:

\begin{equation}
    \Gamma(s) = \int_0^{+\infty} x^{s-1} e^{-x} dx \quad (s > 0)
\end{equation}

\noindent \textbf{注:} 积分变量为 $x$,形式必须为 $x$ 的幂次乘以 $e^{-x}$。

\section*{二、核心性质}

以下性质是解题(尤其是广义积分计算)的关键,需熟练记忆。

\subsection*{1. 递推公式}
这是 $\Gamma$ 函数最基本的性质,用于降阶:
\[
    \Gamma(s+1) = s\Gamma(s)
\]

\subsection*{2. 阶乘关系}
当 $n$ 为正整数时,$\Gamma$ 函数与阶乘有如下关系:
\[
    \Gamma(n+1) = n!
\]
\[
    \Gamma(n) = (n-1)!
\]

\subsection*{3. 重要特殊值}
\begin{itemize}
    \item \textbf{整数点:}
    \[
        \Gamma(1) = 0! = 1
    \]
    \item \textbf{半整数点(高斯积分核心):}
    \[
        \Gamma\left(\frac{1}{2}\right) = \sqrt{\pi}
    \]
\end{itemize}

\subsection*{4. 半整数推导(由性质 1 与 3 推导)}
利用递推公式 $\Gamma(s+1) = s\Gamma(s)$:
\[
    \Gamma\left(\frac{3}{2}\right) = \frac{1}{2}\Gamma\left(\frac{1}{2}\right) = \frac{\sqrt{\pi}}{2}
\]
\[
    \Gamma\left(\frac{5}{2}\right) = \frac{3}{2} \cdot \frac{1}{2}\Gamma\left(\frac{1}{2}\right) = \frac{3\sqrt{\pi}}{4}
\]

\section*{三、常用积分公式结论}

基于 $\Gamma$ 函数定义的推广公式(解题神器):

\[
    \int_0^{+\infty} x^n e^{-kx} dx = \frac{\Gamma(n+1)}{k^{n+1}} = \frac{n!}{k^{n+1}} \quad (n \in \mathbb{N}, k > 0)
\]

\newpage
\chapter{标注小技巧}\label{ch:crossref}
\section*{第一部分:曲线积分 (Curve Integrals)}

\subsection*{一、第一类曲线积分(对弧长的曲线积分)}
% 使用 tabularx 环境,使表格宽度适应文本宽度
% 列定义:p{固定宽度} 和 X{自适应宽度}
\begin{table}[htbp]
    \centering
    \begin{tabularx}{\textwidth}{p{2.5cm} p{3.5cm} X p{4.5cm}}
    \toprule
    \textbf{一级分类} & \textbf{具体题型/场景} & \textbf{核心解法/步骤} & \textbf{注意/技巧} \\
    \midrule
    
    对称性与两点法 & 积分曲线L关于坐标轴(如x轴、y轴)对称。 & 根据被积函数 $f(x,y)$ 关于对应变量的奇偶性判断积分结果。
    \begin{enumerate}
        \item 若为奇函数,积分为0。
        \item 若为偶函数,积分等于在单侧曲线上积分的2倍。
    \end{enumerate} & \textbf{【两点法】}(简化计算/推导结论)
    \newline 用于快速判断对称性,无需记忆公式:
    \begin{enumerate}
        \item 写出点P关于对称轴的对称点P'。
        \item 写出P和P'对应的被积分式(\textbf{注意:第一类积分中ds不变})。
        \item 两式相加。若和为零,则总积分为0。
    \end{enumerate} \\
    \midrule
    
    对称性与两点法 & \textbf{特殊对称性(变量对称性)}:积分曲线L关于直线 $y=x$ 对称。 & 利用变量轮换性质:
    \newline $\int_{L}f(x,y)ds=\int_{L}f(y,x)ds$。 & \textbf{【注】}:若曲线方程 $F(x,y)=0$ 满足 $F(x,y)=F(y,x)$ 或 $F(x,y)=-F(y,x)$,则曲线关于 $y=x$ 对称。 \\
    \midrule
    
    计算方法 & L由显函数 $y=g(x)$, $a\le x\le b$ 给出。 & 化为定积分:
    \newline $\int_{L}f(x,y)ds=\int_{a}^{b}f(x,g(x))\sqrt{1+[g^{\prime}(x)]^{2}}dx$。 & 核心是计算弧微分 $ds$。 \\
    \midrule
    
    计算方法 & L由参数方程 $\begin{cases}x=x(t)\\ y=y(t)\end{cases}$, $\alpha\le t\le\beta$ 给出。 & 化为定积分:
    \newline $\int_{L}f(x,y)ds=\int_{\alpha}^{\beta}f(x(t),y(t))\sqrt{[x^{\prime}(t)]^{2}+[y^{\prime}(t)]^{2}}dt$。 & 适用于复杂的平面曲线或空间曲线。 \\
    \midrule
    
    计算方法 & L由极坐标 $r=r(\theta)$, $\alpha\le\theta\le\beta$ 给出。 & 化为定积分:
    \newline $\int_{L}f(x,y)ds=\int_{\alpha}^{\beta}f(r\cos\theta,r\sin\theta)\sqrt{r^{2}+r^{\prime2}}d\theta$。 & 适用于积分路径为圆或与圆相关的曲线。 \\
    \midrule
    
    简化运算技巧 & 被积表达式复杂。 & \textbf{代入曲线方程简化运算}:将被积函数中的部分表达式用曲线方程进行替换或化简(例如利用 $x^2+y^2=R^2$)。 & 对称性与代入方程简化通常结合运用。 \\
    
    \bottomrule
    \end{tabularx}
\end{table}

\subsection*{二、第二类曲线积分(对坐标的曲线积分)}
\begin{table}[htbp]
    \centering
    \begin{tabularx}{\textwidth}{p{2.5cm} p{3.5cm} X p{4.5cm}}
    \toprule
    \textbf{一级分类} & \textbf{具体题型/场景} & \textbf{核心解法/步骤} & \textbf{注意/技巧} \\
    \midrule
    
    对称性与两点法 & 有向积分曲线L关于坐标轴对称。 & 根据被积函数关于对应变量的奇偶性以及积分方向判断积分结果。 & \textbf{【两点法】}(简化计算/推导结论)
    \begin{enumerate}
        \item 写出点P关于对称轴的对称点P'。
        \item 写出P和P'对应的被积分式。\textbf{关键注意}:由于曲线有向,对称点的微分项可能会变号(例如关于x轴对称时,若路径方向相反,dx可能变为-dx)。
        \item 两式相加。若和为零,则总积分为0。
    \end{enumerate} \\
    \midrule
    
    计算方法 & L由显函数 $y=g(x)$ 给出,起点对应x=a,终点对应x=b。 & 化为定积分:
    \newline $\int_{L}Pdx+Qdy=\int_{a}^{b}[P(x,g(x))+Q(x,g(x))g^{\prime}(x)]dx$。 & 注意积分上下限必须与曲线方向一致。 \\
    \midrule
    
    计算方法 & L由参数方程给出,起点对应$t=\alpha$,终点对应$t=\beta$。 & 化为定积分:
    \newline $\int_{L}Pdx+Qdy=\int_{\alpha}^{\beta}[P(x(t),y(t))x^{\prime}(t)+Q(x(t),y(t))y^{\prime}(t)]dt$。 & \textbf{【曲线方程参数化】}:常用于处理空间曲线积分(如两个曲面的交线,例如柱面与平面相交)。需要根据题意正确设定参数方程和方向。 \\
    \midrule
    
    计算方法 & L由极坐标 $r=r(\theta)$ 给出,起点对应$\theta=\alpha$,终点对应$\theta=\beta$。 & 化为定积分(根据讲义公式):
    \newline $\int_{L}Pdx+Qdy=\int_{\alpha}^{\beta}\{P[r(\theta)\cos\theta]^{\prime}+Q[r(\theta)\sin\theta]^{\prime}\}d\theta$。 & 注意将x,y代入后对 $\theta$ 求导。 \\
    
    \bottomrule
    \end{tabularx}
\end{table}

\subsection*{三、格林公式及其应用技巧}
\begin{table}[htbp]
    \centering
    \begin{tabularx}{\textwidth}{p{3cm} p{3cm} X p{4.5cm}}
    \toprule
    \textbf{一级分类} & \textbf{具体题型/场景} & \textbf{核心解法/步骤} & \textbf{注意/技巧} \\
    \midrule
    
    格林公式 & 平面闭区域D上的第二类曲线积分。P, Q在D上具有一阶连续偏导数。 & 将沿正向边界L的曲线积分转化为二重积分:
    \newline $\oint_{L}Pdx+Qdy=\iint_{D}(\frac{\partial Q}{\partial x}-\frac{\partial P}{\partial y})dxdy$。 & L必须是D的\textbf{所有}正向(通常是逆时针)边界曲线。 \\
    \midrule
    
    格林公式应用技巧一:补线 & 曲线L不是闭曲线,且直接计算不易。 & 
    \begin{enumerate}
        \item 作辅助路径 $\Gamma$,使L与$\Gamma$构成闭曲线,围成区域D。
        \item 利用格林公式计算闭曲线积分。
        \item 原积分 = 格林公式结果 - 辅助路径上的积分。
    \end{enumerate} & 
    \begin{enumerate}
        \item 要求D内无奇点。
        \item \textbf{【$\Gamma$的选取】}:使得在 $\Gamma$ 上的积分容易计算,一般取折线或线段。注意构造时保持正向(逆时针)。
    \end{enumerate} \\
    \midrule
    
    格林公式应用技巧二:补圈挖点 & 闭曲线L所围区域内有奇点(P, Q偏导数不连续或无定义)。 & 
    \begin{enumerate}
        \item 在L内部,补上一条包含奇点的小闭曲线 $\Gamma$(挖去奇点),使L和$\Gamma$之间的区域D不含奇点。
        \item 在D上应用格林公式。
        \item $\int_{L}Pdx+Qdy=\int_{\Gamma}Pdx+Qdy+\iint_{D}(\frac{\partial Q}{\partial x}-\frac{\partial P}{\partial y})dxdy$。
    \end{enumerate} & \textbf{【$\Gamma$的选取】}:使得在 $\Gamma$ 上的积分容易计算。通常选取能消去分母的曲线。例如被积式为 $\frac{xdy-ydx}{4x^{2}+y^{2}}$,可取 $\Gamma$ 为 $4x^{2}+y^{2}=1$。注意L和-$\Gamma$是D的正向边界。 \\
    
    \bottomrule
    \end{tabularx}
\end{table}

\subsection*{四、路径无关问题与斯托克斯公式}
\begin{table}[htbp]
    \centering
    \begin{tabularx}{\textwidth}{p{2.5cm} p{3.5cm} X p{4.5cm}}
    \toprule
    \textbf{一级分类} & \textbf{具体题型/场景} & \textbf{核心解法/步骤} & \textbf{注意/技巧} \\
    \midrule
    
    路径无关问题 & 判断积分与路径无关,或利用该性质求积分、求参数/函数。 & 在单连通域G内,若P, Q有连续偏导数,则路径无关等价于以下条件(满足其一即可):
    \begin{enumerate}
        \item 沿G内任意闭曲线积分为0。
        \item $Pdx+Qdy$ 是某函数 $U(x,y)$ 的全微分 $dU(x,y)$。
        \item $\frac{\partial P}{\partial y}=\frac{\partial Q}{\partial x}$ 在G内处处成立。
    \end{enumerate} & 若积分与路径无关,可以直接将积分写为 $\int_{(x_{1},y_{1})}^{(x_{2},y_{2})}$,并可选择最简单的路径(如折线)进行计算,或求出原函数U。 \\
    \midrule
    
    斯托克斯公式 & 空间有向闭曲线 $\Gamma$ 上的第二类曲线积分。 & 将曲线积分转化为以 $\Gamma$ 为边界的有向曲面 $\Sigma$ 上的曲面积分(计算旋度):
    \newline $\oint_{\Gamma}Pdx+Qdy+Rdz=\iint_{\Sigma}\begin{vmatrix}dydz&dxdz&dxdy\\ \frac{\partial}{\partial x}&\frac{\partial}{\partial y}&\frac{\partial}{\partial z}\\ P&Q&R\end{vmatrix}$ & 
    \begin{enumerate}
        \item $\Gamma$ 的正向与 $\Sigma$ 的侧符合右手规则。
        \item P, Q, R需在 $\Sigma$ 上具有一阶连续偏导数。
    \end{enumerate} \\
    \midrule
    
    斯托克斯公式应用:补线 & 空间曲线 $\Gamma$ 不是闭曲线。 & 补线使之成为闭曲线,然后应用斯托克斯公式,再减去补线上积分。 & 类似于格林公式的补线技巧,但应用于空间曲线。 \\
    
    \bottomrule
    \end{tabularx}
\end{table}


\newpage
\section*{第二部分:曲面积分 (Surface Integrals)}

\subsection*{一、第一类曲面积分(对面积的曲面积分)}
\begin{table}[htbp]
    \centering
    \begin{tabularx}{\textwidth}{p{2.5cm} p{3.5cm} X p{4.5cm}}
    \toprule
    \textbf{一级分类} & \textbf{具体题型/场景} & \textbf{核心解法/步骤} & \textbf{注意/技巧} \\
    \midrule
    
    对称性与两点法 & 积分曲面 $\Sigma$ 关于坐标平面(如xOy面)对称。 & 根据被积函数关于对应变量(如z)的奇偶性判断积分结果(积分为0或2倍)。 & \textbf{【两点法】}(判断第一类曲面积分对称性)
    \begin{enumerate}
        \item 写出点P关于对称面的对称点P'。
        \item 写出P和P'对应的被积分式(\textbf{注意:dS不变})。
        \item 两式相加。若和为零,则总积分为0。
    \end{enumerate} \\
    \midrule
    
    对称性与两点法 & \textbf{特殊对称性(变量对称性)}:积分曲面 $\Sigma$ 关于特定平面对称(如 $y=x$)。 & 利用变量轮换性质:
    \newline $\iint_{\Sigma}f(x,y,z)dS=\iint_{\Sigma}f(y,x,z)dS$。 & \textbf{【注】}:若曲面方程 $F(x,y,z)=0$ 满足 $F(x,y,z)=F(y,x,z)$,则曲面关于 $y=x$ 对称。 \\
    \midrule
    
    计算方法:投影法 & 积分曲面由 $z=z(x,y)$ 给出,投影为 $D_{xy}$。 & \textbf{【投影法】}(将曲面积分化为二重积分):
    \newline $\iint_{\Sigma}f(x,y,z)dS=\iint_{D_{xy}}f(x,y,z(x,y))\sqrt{1+z_{x}^{2}+{z_{y}}^{2}}dxdy$。 & 核心是计算面积微元 $dS$。同样可以向yOz面或xOz面投影。 \\
    \midrule
    
    特殊应用 & 计算特殊曲面(如柱面被截)的面积。 & \textbf{【弧微分求曲面面积】}:利用弧微分的思想计算曲面面积。 & 适用于计算柱面等侧面积。 \\
    
    \bottomrule
    \end{tabularx}
\end{table}

\subsection*{二、第二类曲面积分(对坐标的曲面积分)}
\begin{table}[htbp]
    \centering
    \begin{tabularx}{\textwidth}{p{3cm} p{3cm} X p{4.5cm}}
    \toprule
    \textbf{一级分类} & \textbf{具体题型/场景} & \textbf{核心解法/步骤} & \textbf{注意/技巧} \\
    \midrule
    
    对称性与两点法 & 积分曲面 $\Sigma$(有向)关于坐标平面对称。 & 根据被积函数的奇偶性以及曲面方向判断积分结果。 & \textbf{【两点法】}(判断第二类曲面积分对称性)
    \begin{enumerate}
        \item 写出点P关于对称面的对称点P'。
        \item 写出P和P'对应的被积分式。\textbf{关键注意}:由于曲面有向,对称点的微分项可能会变号(例如关于xOy面对称时,上侧的dxdy在下侧可能对应-dxdy)。
        \item 两式相加。若和为零,则总积分为0。
    \end{enumerate} \\
    \midrule
    
    计算方法一:投影法 & 积分曲面由显函数给出(如 $z=z(x,y)$)。 & \textbf{【投影法】}(分别计算三个分量):
    \newline 分别计算 $\iint_{\Sigma}P dydz, \iint_{\Sigma}Q dxdz, \iint_{\Sigma}R dxdy$。
    \newline 以R为例:$\iint_{\Sigma}Rdxdy=\pm\iint_{D_{xy}}R(x,y,z(x,y))dxdy$。 & \textbf{【符号判断】}:
    \newline 当曲面法向量与对应轴(如z轴)正向夹角为锐角时取“+”,钝角时取“-”。 \\
    \midrule
    
    计算方法二:利用两类曲面积分的联系 & 通用计算方法,特别是曲面法向量容易求得时。 & 将第二类曲面积分转化为第一类曲面积分:
    \newline $\iint_{\Sigma}Pdydz+Qdxdz+Rdxdy=\iint_{\Sigma}(P\cos\alpha+Q\cos\beta+R\cos\gamma)dS$。
    \newline 其中 $\mathbf{n}=(\cos\alpha,\cos\beta,\cos\gamma)$ 是曲面指定侧的单位法向量。 & 
    \begin{enumerate}
        \item 关键是计算单位法向量 $\mathbf{n}$(讲义提供了显函数和隐函数的计算公式)。
        \item \textbf{【正负号判断】}:计算出的法向量需要通过图像判断其与指定侧的方向是相同(取正号)还是相反(取负号)。
    \end{enumerate} \\
    \midrule
    
    计算方法三:合一投影法 & 积分曲面由显函数(如 $z=z(x,y)$)或隐函数给出。 & \textbf{【合一投影法】}(将三个分量同时投影到一个坐标面):
    \newline 若投影到$D_{xy}$(设 $\Sigma: z=z(x,y)$):
    \newline $\iint_{\Sigma}Pdydz+Qdxdz+Rdxdy=\iint_{D_{xy}}[P\cdot(-z_{x})+Q\cdot(-z_{y})+R\cdot1]dxdy$。 & 这个方法用起来和方法一差不多。对于隐函数 $F(x,y,z)=0$,可以用偏导数的比值代换。 \\
    
    \bottomrule
    \end{tabularx}
\end{table}

\subsection*{三、高斯公式及其应用技巧}
\begin{table}[htbp]
    \centering
    \begin{tabularx}{\textwidth}{p{3cm} p{3cm} X p{4.5cm}}
    \toprule
    \textbf{一级分类} & \textbf{具体题型/场景} & \textbf{核心解法/步骤} & \textbf{注意/技巧} \\
    \midrule
    
    计算方法四:高斯公式 & 空间闭区域 $\Omega$ 上的曲面积分。P, Q, R在 $\Omega$ 上具有一阶连续偏导数。 & \textbf{【高斯公式】}(将闭曲面积分转化为三重积分):
    \newline $\iint_{\Sigma}Pdydz+Qdzdx+Rdxdy = \iiint_{\Omega}(\frac{\partial P}{\partial x}+\frac{\partial Q}{\partial y}+\frac{\partial R}{\partial z})dv$。 & $\Sigma$ 必须是 $\Omega$ 的所有外侧边界曲面。 \\
    \midrule
    
    高斯公式应用技巧一:补面 & 曲面 $\Sigma$ 不是闭曲面,且直接计算不易。 & 
    \begin{enumerate}
        \item 作辅助曲面 $\Sigma'$,使 $\Sigma+\Sigma'$ 构成闭曲面,围成区域 $\Omega$。
        \item 利用高斯公式计算闭曲面积分。
        \item 原积分 = 高斯公式结果 - 辅助曲面 $\Sigma'$ 上的积分。
    \end{enumerate} & 
    \begin{enumerate}
        \item $\Sigma$ 与 $\Sigma'$ 的方向应构成 $\Omega$ 的外侧。要求$\Omega$内无奇点。
        \item \textbf{【$\Sigma'$的选取】}:使得在 $\Sigma'$ 上的积分容易计算,一般取平面。
    \end{enumerate} \\
    \midrule
    
    高斯公式应用技巧二:挖点 & 闭曲面 $\Sigma$ 所围区域内有奇点。 & 
    \begin{enumerate}
        \item 在 $\Sigma$ 内部,补上一个包含奇点的小闭曲面 $\Sigma'$(挖去奇点),使 $\Sigma$ 和 $\Sigma'$ 之间的区域 $\Omega$ 不含奇点。
        \item 在 $\Omega$ 上应用高斯公式。
        \item 原积分 = $\Sigma'$ 上的积分 + 高斯公式结果。
    \end{enumerate} & \textbf{【$\Sigma'$的选取】}:使得在 $\Sigma'$ 上的积分容易计算。通常选取能消去分母的曲面。例如被积式分母为 $(x^{2}+y^{2}+z^{2})^{3/2}$,可取 $\Sigma'$ 为小球面 $x^{2}+y^{2}+z^{2}=\epsilon^{2}$。注意方向的配置。 \\
    
    \bottomrule
    \end{tabularx}
\end{table}
\newpage

\chapter{算子法求解微分方程特解的“四大杀手锏”规则}
\begin{center}
    
\end{center}

\section{通用共振处理逻辑(求导法则)}
当直接代入导致分母 $L(D) = 0$ 时,使用“分母求导”公式。

\textbf{公式:}
\[
\frac{1}{L(D)} f(x) = x \cdot \frac{1}{L'(D)} f(x)
\]

\begin{itemize}
    \item \textbf{操作口诀:} 分母为零别慌张,\textbf{分子乘 $x$,分母求导},然后再试一次。如果还为 0,就再乘 $x$ 再求导。
\end{itemize}

\section{规则一:指数函数 $f(x) = e^{\lambda x}$ (代入规则)}

\textbf{1. 常规情况 ($L(\lambda) \neq 0$):}
直接用 $\lambda$ 代替 $D$。
\[
y^* = \frac{1}{L(\lambda)} e^{\lambda x}
\]

\textbf{2. 共振情况 ($L(\lambda) = 0$):}
使用求导法则。
\[
y^* = x \cdot \frac{1}{L'(\lambda)} e^{\lambda x}
\]
(注:若 $L'(\lambda)$ 仍为 0,则继续求导变为 $x^2 \cdot \frac{1}{L''(\lambda)} e^{\lambda x}$)

\begin{itemize}
    \item \textbf{例:} $(D-2)^2 y = e^{2x}$
    \begin{itemize}
        \item \textbf{分析:} 代入 $D=2$ 分母为 0。
        \item \textbf{一次求导:} $x \cdot \frac{1}{2(D-2)} e^{2x}$ (再次代入 $D=2$ 仍为 0)
        \item \textbf{二次求导:} $x^2 \cdot \frac{1}{2} e^{2x} = \frac{1}{2}x^2 e^{2x}$
    \end{itemize}
\end{itemize}

\section{规则二:三角函数 $f(x) = \sin(\omega x)$ 或 $\cos(\omega x)$ (平方代入规则)}

\textbf{1. 常规情况 ($L(-\omega^2) \neq 0$):}
如果算子 $L(D)$ 中只含有 $D^2$ 的项(或者能化简出 $D^2$),可以用 $-\omega^2$ 代替 $D^2$。

\textbf{2. 共振情况 ($L(-\omega^2) = 0$):}
使用求导法则(注意是对 $D$ 求导)。
\[
y^* = x \cdot \frac{1}{L'(D)} \sin(\omega x)
\]
通常求导后分母会变成含 $D$ 的一阶式,此时利用 $\frac{1}{D} = \int dx$ 进行积分。

\begin{itemize}
    \item \textbf{例:} $(D^2 + 4)y = \sin(2x)$
    \begin{itemize}
        \item \textbf{分析:} 代入 $D^2 = -4$ 分母为 0。
        \item \textbf{分母求导:} $(D^2+4)' = 2D$
        \item \textbf{应用法则:} 
        \[
        y^* = x \cdot \frac{1}{2D} \sin(2x) = \frac{x}{2} \int \sin(2x) \, dx = -\frac{x}{4}\cos(2x)
        \]
    \end{itemize}
\end{itemize}

\section{规则三:多项式 $f(x) = P_n(x)$ (级数展开规则)}

利用泰勒级数展开,将逆算子 $\frac{1}{L(D)}$ 展开成 $D$ 的幂级数。

\textbf{★ 关键:如何判断展开阶数?}
看多项式 $P_n(x)$ 的最高次数 $n$,级数展开只需\textbf{保留到 $D^n$ 项}。
\begin{itemize}
    \item \textbf{原理:} $D^{n+1} P_n(x) = 0$(更高阶导数为 0,直接舍弃)。
    \item \textbf{速判:} 
    \begin{itemize}
        \item 针对 $x$ (1次) $\rightarrow$ 保留到 $D$。
        \item 针对 $x^2$ (2次) $\rightarrow$ 保留到 $D^2$。
    \end{itemize}
\end{itemize}

\textbf{1. 常规情况(有常数项):}
直接按泰勒级数展开。

\textbf{2. 共振情况(无常数项):}
当算子最低次项为 $D^k$ 时(对应 0 特征根),使用\textbf{提公因式法}。

\begin{itemize}
    \item \textbf{步骤:}
    \begin{enumerate}
        \item 提出分母中最低次的 $D^k$。
        \item 对剩下的部分进行级数展开(保留至 $D^n$)。
        \item 最后进行 $k$ 次积分(因为 $\frac{1}{D}$ 等价于积分)。
    \end{enumerate}
    \item \textbf{例:} $(D^2 - D)y = x$
    \begin{itemize}
        \item \textbf{推导:} 提出 $D$,剩下 $\frac{1}{D-1}$ 对 $x$ (1次) 展开至 $D$ 项。
        \[
        y^* = \frac{1}{D(D-1)} x = \frac{1}{D} \left[ -(1+D) \right] x = -\frac{1}{D}(x+1) = -(\frac{x^2}{2} + x)
        \]
    \end{itemize}
\end{itemize}

\section{规则四:移位定理(Shift Theorem)}
当 $f(x)$ 是指数函数与其他函数的乘积时,可以将指数函数“移”到算子左边,同时算子中的 $D$ 变为 $D+\lambda$。

\textbf{公式:}
\[
\frac{1}{L(D)} [e^{\lambda x} v(x)] = e^{\lambda x} \frac{1}{L(D+\lambda)} v(x)
\]

\begin{itemize}
    \item \textbf{应用场景:} 处理乘积形式,或者用于化简复杂的共振。
    \item \textbf{与求导法则联动:} 如果移位后,剩下的算子对 $v(x)$ 处理时依然分母为 0,则继续对剩下的部分使用求导法则。
    \item \textbf{例:} $(D-1)y = e^x$
    \begin{itemize}
        \item \textbf{方法 A (移位法):} 看作 $e^x \cdot 1$。
        \[
        y^* = e^x \frac{1}{(D+1)-1} \cdot 1 = e^x \frac{1}{D} \cdot 1 = x e^x
        \]
        \item \textbf{方法 B (直接求导法):} 直接看 $e^x$,代入 $D=1$ 分母为 0。
        \[
        y^* = x \cdot \frac{1}{(D-1)'} e^x = x \cdot \frac{1}{1} e^x = x e^x
        \]
    \end{itemize}
\end{itemize}

\section{总结:什么时候用什么?}
\begin{itemize}
    \item 看到 $e^{\lambda x}$ $\rightarrow$ \textbf{直接代}。分母为 0? $\rightarrow$ \textbf{分子乘 $x$,分母求导}。
    \item 看到 $\sin / \cos$ $\rightarrow$ \textbf{代 $D^2$}。分母为 0? $\rightarrow$ \textbf{分子乘 $x$,分母求导}(然后积分)。
    \item 看到 $e^{\lambda x} \cdot \dots$ $\rightarrow$ \textbf{先移位},把 $e$ 移走,再处理剩下的。
    \item 看到多项式 $\rightarrow$ \textbf{展开}(保留至 $D^n$)。没常数项? $\rightarrow$ \textbf{提 $D$ 积分}。
\end{itemize}
\newpage
\chapter{考研数学一常用泰勒展开公式汇总}
\author{}
\date{}

在考研数学中,泰勒展开是极限、级数等题目的重要工具。以下是必须熟练掌握的常用展开式:

\section{基本初等函数的泰勒展开}

\subsection{指数函数}
\begin{equation}
e^x = 1 + x + \frac{x^2}{2!} + \frac{x^3}{3!} + \cdots + \frac{x^n}{n!} + o(x^n)
\end{equation}
收敛域:$(-\infty, +\infty)$

\subsection{对数函数}
\begin{equation}
\ln(1+x) = x - \frac{x^2}{2} + \frac{x^3}{3} - \frac{x^4}{4} + \cdots + (-1)^{n-1}\frac{x^n}{n} + o(x^n)
\end{equation}
收敛域:$(-1, 1]$

\subsection{三角函数}
\begin{equation}
\sin x = x - \frac{x^3}{3!} + \frac{x^5}{5!} - \frac{x^7}{7!} + \cdots + (-1)^n\frac{x^{2n+1}}{(2n+1)!} + o(x^{2n+1})
\end{equation}

\begin{equation}
\cos x = 1 - \frac{x^2}{2!} + \frac{x^4}{4!} - \frac{x^6}{6!} + \cdots + (-1)^n\frac{x^{2n}}{(2n)!} + o(x^{2n})
\end{equation}

\begin{equation}
\tan x = x + \frac{x^3}{3} + \frac{2x^5}{15} + \frac{17x^7}{315} + \cdots
\end{equation}
(前几项,$|x| < \frac{\pi}{2}$)

\subsection{反三角函数}
\begin{equation}
\arcsin x = x + \frac{x^3}{6} + \frac{3x^5}{40} + \frac{15x^7}{336} + \cdots
\end{equation}
收敛域:$|x| \leq 1$

\begin{equation}
\arctan x = x - \frac{x^3}{3} + \frac{x^5}{5} - \frac{x^7}{7} + \cdots + (-1)^n\frac{x^{2n+1}}{2n+1} + o(x^{2n+1})
\end{equation}
收敛域:$|x| \leq 1$

\section{幂函数的展开}
\begin{equation}
(1+x)^{\alpha} = 1 + \alpha x + \frac{\alpha(\alpha-1)}{2!}x^2 + \frac{\alpha(\alpha-1)(\alpha-2)}{3!}x^3 + \cdots
\end{equation}
收敛域:$|x| < 1$

\subsection{特殊情形}
\begin{itemize}
\item $(1+x)^{-1} = 1 - x + x^2 - x^3 + x^4 - \cdots$
\item $(1-x)^{-1} = 1 + x + x^2 + x^3 + x^4 + \cdots$
\item $(1+x)^{1/2} = 1 + \frac{x}{2} - \frac{x^2}{8} + \frac{x^3}{16} - \cdots$
\end{itemize}

\newpage
% --- 你的内容 ---
\chapter{无穷级数}
\section{常数项级数}
\subsection{常数项方程}

级数$\sum_{n=1}^{\infty} u_n$ 的前 $n$ 项和
\begin{equation}\tag{11.1}
S_n=\sum_{k=1}^{n}u_k
  =u_1+u_2+\cdots+u_n,\qquad (n=1,2,\ldots)
\end{equation}
称为无穷级数的部分和。

若数项级数 $\sum_{n=1}^{\infty} u_n$ 的部分和数列 $\{S_n\}$ 的极限
$\lim_{n\to\infty}S_n$ 存在,则称级数 $\sum_{n=1}^{\infty} u_n$ 收敛;
否则称级数 $\sum_{n=1}^{\infty} u_n$ 发散。当级数 $\sum_{n=1}^{\infty} u_n$ 收敛时,称极限值 $\lim_{n\to\infty}S_n$ 为此级数的和,
即
\[
\sum_{n=1}^{\infty} u_n
  = \lim_{n\to\infty} S_n
  = \lim_{n\to\infty}\sum_{k=1}^{n} u_k
  = S .
\]

% --- 我帮你添加的“压缩映射”部分 ---

\section{压缩映射原理做题步骤}

\subsection{核心思想}
对于形如 $x_{n+1}=f(x_{n})$ 的数列,如果能证明函数 $f(x)$ 在某个区间上是一个“压缩映射”,即 $|f'(x)| \le k < 1$,那么数列 $\{x_n\}$ 必定会收敛到 $f(x)$ 的不动点 $a$(即 $f(a)=a$)。

这种方法的本质是“先斩后奏”:先假设极限 $a$ 存在并求出它,然后再证明数列确实收敛到 $a$。

\subsection{解题步骤}
\begin{enumerate}
    % --- 第 1 步 ---
    \item \textbf{【先斩后奏,求出不动点 $a$】}
    \begin{itemize}
        \item 假设 $\lim_{n \to \infty} x_n = a$ 存在。
        \item 在递推公式 $x_{n+1} = f(x_n)$ 两边同时取极限,得到 $a = f(a)$。
        \item 解此方程,求出不动点 $a$ 的值。(注意:有时需要根据 $x_n$ 的有界性,从多个解中筛选出真正的极限值)。
    \end{itemize}

    % --- 第 2 步 ---
    \item \textbf{【构造压缩形式】}
    \begin{itemize}
        \item 考察 $x_{n+1}$ 与 $a$ 之间的距离,即考察绝对值 $|x_{n+1} - a|$。
        \item 将 $x_{n+1} = f(x_n)$ 和 $a = f(a)$ 代入,得到:
        $$ |x_{n+1} - a| = |f(x_n) - f(a)| $$
    \end{itemize}
    
    % --- 第 3 步 ---
    \item \textbf{【应用拉格朗日中值定理 (MVT)】}
    \begin{itemize}
        \item 对 $|f(x_n) - f(a)|$ 应用拉格朗日中值定理,得到:
        $$ f(x_n) - f(a) = f'(\xi_n)(x_n - a) $$
        其中 $\xi_n$ 位于 $x_n$ 和 $a$ 之间。
        \item 将此代回第2步的等式,得到:
        $$ |x_{n+1} - a| = |f'(\xi_n)| \cdot |x_n - a| $$
    \end{itemize}
    
    % --- 第 4 步 ---
    \item \textbf{【确定压缩常数 $k$】(关键步骤)}
    \begin{itemize}
        \item 证明存在一个 \textbf{常数 $k$},满足 $0 < k < 1$,使得对于所有的 $n$,都有 $|f'(\xi_n)| \le k$。
        \item \textbf{常用技巧:} 利用题目条件(如“$f(x)$ 有连续导数”)和闭区间上连续函数的最值定理,找到 $|f'(x)|$ 在相关区间上的最大值,并证明该最大值 $k$ 严格小于 1。
    \end{itemize}
    
    % --- 第 5 步 ---
    \item \textbf{【递推与夹逼准则】}
    \begin{itemize}
        \item 得到核心不等式:
        $$ |x_{n+1} - a| \le k |x_n - a| $$
        \item 反复应用此不等式:
        $$ |x_{n+1} - a| \le k |x_n - a| \le k^2 |x_{n-1} - a| \le \dots \le k^n |x_1 - a| $$
        \item 因为 $0 < k < 1$,所以 $\lim_{n \to \infty} k^n = 0$。
        \item 根据夹逼准则(Squeeze Theorem):
        $$ 0 \le \lim_{n \to \infty} |x_{n+1} - a| \le \lim_{n \to \infty} \left( k^n |x_1 - a| \right) = 0 $$
        \item 因此,$\lim_{n \to \infty} |x_{n+1} - a| = 0$,这等价于 $\lim_{n \to \infty} x_n = a$。
    \end{itemize}
\end{enumerate}

\newpage
\chapter{场论四大核心概念}

\section{核心工具:倒三角算子 (\texorpdfstring{$\nabla$}{nabla})}

\begin{definition}[倒三角算子 $\nabla$]
所有场论运算的核心都是这个算子。请把它看作一个\textbf{“带求导功能的向量”}。
\begin{itemize}
    \item \textbf{符号}:$\nabla$ 
    \item \textbf{定义}:$\displaystyle \nabla = \left( \frac{\partial}{\partial x}, \ \frac{\partial}{\partial y}, \ \frac{\partial}{\partial z} \right)$
    \item \textbf{心法}:像操作向量一样操作它(点积、叉积、数乘),但它的动作是“求导”而不是简单的“数值乘法”。
\end{itemize}
\end{definition}

\section{深度解析与公式}

% 这里的标题去掉了原本的 "1.",直接显示核心概念,配合 definition 环境更突出
\begin{definition}[梯度 $\operatorname{grad} u$]
\begin{itemize}
    \item \textbf{定义}:标量场 $u(x,y,z)$ 在某点的最大变化率方向。
    \item \textbf{公式}:
    \[
        \operatorname{grad} u = \nabla u = \left( \frac{\partial u}{\partial x}, \ \frac{\partial u}{\partial y}, \ \frac{\partial u}{\partial z} \right)
    \]
    \item \textbf{性质}:
    \begin{itemize}
        \item 梯度垂直于等值面(法向量 $\vec{n} = \nabla u$)。
        \item 模长 $|\nabla u|$ 是该点最大的变化率数值。
    \end{itemize}
\end{itemize}
\end{definition}

\begin{definition}[方向导数 $\frac{\partial u}{\partial l}$]
\begin{itemize}
    \item \textbf{定义}:函数 $u$ 沿\textbf{指定方向} $\vec{l}$ 的变化率。
    \item \textbf{公式(核心考点)}:
    \[
        \frac{\partial u}{\partial l} = \nabla u \cdot \vec{e}_l = |\nabla u| \cos\theta
    \]
    \begin{itemize}
        \item $\vec{e}_l$:方向 $\vec{l}$ 的\textbf{单位向量} \textcolor{red}{(做题必坑点:记得单位化!)}。
        \item $\theta$:梯度与方向 $\vec{l}$ 的夹角。
    \end{itemize}
    \item \textbf{三者关系}:
    \begin{itemize}
        \item $\theta = 0$(同向):方向导数最大($= |\nabla u|$)。
        \item $\theta = \pi$(反向):方向导数最小($= -|\nabla u|$)。
        \item $\theta = \frac{\pi}{2}$(垂直/沿等高线):方向导数 $= 0$。
    \end{itemize}
\end{itemize}
\end{definition}

\begin{definition}[散度 $\operatorname{div} \vec{A}$]
\begin{itemize}
    \item \textbf{定义}:通量对体积的变化率,描述场的“发散”程度。
    \item \textbf{公式(点积法则)}:
    设 $\vec{A} = (P, Q, R)$,
    \[
        \operatorname{div} \vec{A} = \nabla \cdot \vec{A} = \frac{\partial P}{\partial x} + \frac{\partial Q}{\partial y} + \frac{\partial R}{\partial z}
    \]
    \item \textbf{物理意义}:$>0$ 发散(有源);$<0$ 汇聚(有汇);$=0$ 无源场。
\end{itemize}
\end{definition}

\begin{definition}[旋度 $\operatorname{rot} \vec{A}$ / $\operatorname{curl} \vec{A}$]
\begin{itemize}
    \item \textbf{定义}:环量对面密度的极限,描述场的“旋转”强度。
    \item \textbf{公式(叉积法则 - 背行列式)}:
    \[
        \operatorname{rot} \vec{A} = \nabla \times \vec{A} = \begin{vmatrix} \vec{i} & \vec{j} & \vec{k} \\ \frac{\partial}{\partial x} & \frac{\partial}{\partial y} & \frac{\partial}{\partial z} \\ P & Q & R \end{vmatrix}
    \]
    \[
        = \left(\frac{\partial R}{\partial y} - \frac{\partial Q}{\partial z}\right)\vec{i} - \left(\frac{\partial R}{\partial x} - \frac{\partial P}{\partial z}\right)\vec{j} + \left(\frac{\partial Q}{\partial x} - \frac{\partial P}{\partial y}\right)\vec{k}
    \]
\end{itemize}
\end{definition}


\section{两个重要的“恒等于零” (必考)}

% 使用 property 环境,颜色与 definition 不同,层次更丰富
\begin{property}[场论重要恒等式]
\begin{enumerate}
    \item \textbf{梯度的旋度为零}:
    \[ \operatorname{rot}(\operatorname{grad} u) = \nabla \times (\nabla u) = \vec{0} \]
    \textit{记忆}:$\vec{a} \times \vec{a} = \vec{0}$(自己不能绕着自己转)。\\
    \textit{意义}:梯度场(保守场,如重力场、静电场)是无旋的。
    
    \item \textbf{旋度的散度为零}:
    \[ \operatorname{div}(\operatorname{rot} \vec{A}) = \nabla \cdot (\nabla \times \vec{A}) = 0 \]
    \textit{记忆}:$\vec{a} \cdot (\vec{a} \times \vec{b}) = 0$(垂直向量点积为0)。\\
    \textit{意义}:旋涡场(如磁场)没有源头,磁感线是闭合的。
\end{enumerate}
\end{property}
\newpage
行列式:一、 递推公式法 (主要针对三对角行列式)
1. 适用结构:通常是三对角矩阵(即只有主对角线以及主对角线上方和下方的一条对角线有非零元素,其余全为0)。例如:$$\begin{vmatrix}
a & b & 0 & \cdots & 0 \\
c & a & b & \cdots & 0 \\
0 & c & a & \cdots & 0 \\
\vdots & \vdots & \vdots & \ddots & \vdots \\
0 & 0 & 0 & \cdots & a
\end{vmatrix}$$2. 核心步骤:建立递推关系: 按照第一行(或第一列)展开。由于第一行只有两个非零元素,展开后会得到一个 $n-1$ 阶行列式和一个 $n-2$ 阶行列式。通常形式为:$D_n = a D_{n-1} - bc D_{n-2}$。求解递推数列: 得到关系式后,将其视为数列问题求解。3. 求解技巧(图片中提到的方法):方法使用数列技巧


二、 爪形行列式求法 (及镶边行列式)这在图片 [例 1.18]、[例 1.19] 和 [例 1.21] 中出现。1. 适用结构:像“爪子”一样的形状。通常是第一行、第一列以及主对角线有非零元素,其余位置全为0(或者最后一行、最后一列、斜对角线)。$$\begin{vmatrix}
a_1 & a_2 & a_3 & \cdots & a_n \\
b_1 & c_2 & 0 & \cdots & 0 \\
b_2 & 0 & c_3 & \cdots & 0 \\
\vdots & \vdots & \vdots & \ddots & \vdots \\
b_n & 0 & 0 & \cdots & c_n
\end{vmatrix}$$2. 核心解法:直接展开法(图片中的解法):按第一行(或爪子的“柄”所在的行/列)展开。图片 [例 1.18] 展示了展开后会得到 $D_n = a_1 \prod c_i - \sum (\text{各项})$ 的形式。图片总结的公式是:$D_n = (\prod_{i=1}^n c_i) (a_0 - \sum_{i=1}^n \frac{b_i a_i}{c_i})$ (针对特定的 $a_0$ 在左上角的情况)。化为三角行列式(消元法):利用对角线上的元素 $c_i$,将第一列(或第一行)的 $b_i$ 消成 0。这通常是最通用的方法,将“爪子”剪掉,变成上三角或下三角行列式。



线性相关与线性无关:r代表长成的空间维数,列数m是向量的数量,r的维数=m的个数则线性无关,没有多余向量。r<m则有多余的向量线性相关
齐次方程的解也即向量在线性无关时的系数,存在不为零的系数则有非零解,此时线性相关r<m,只存在零解的时候则线性无关r=m


矩阵等价 (Matrix Equivalence):定义:矩阵 $A$ 经过有限次初等变换(行或列都可以)变成矩阵 $B$。充要条件:$r(A) = r(B)$(秩相等)。口诀:只要秩一样,矩阵就等价。向量组等价 (Vector Group Equivalence):定义:向量组 (I) 和向量组 (II) 可以相互线性表示(生成的空间相同)。充要条件:$r(I) = r(II) = r(I, II)$。直观理解:它们“长”在同一个空间里,且张成的范围(维数)一样。

初等行变换的核心性质:保持行空间不变: 行向量组是等价的。保持列向量的线性关系不变: $Ax=0$ 和 $Bx=0$ 同解。破坏列空间: 除非 $P$ 是单位阵,否则列空间通常会变。

公式 $(A^*)^* = |A|^{n-2} A$ 是线性代数中的一个经典证明
利用伴随矩阵的定义(推荐,适用于 $|A| \neq 0$)核心工具:伴随矩阵的基本性质:$M M^* = M^* M = |M| E$(对任意 $n$ 阶矩阵 $M$ 成立)。伴随矩阵的行列式性质:$|A^*| = |A|^{n-1}$。推导步骤:将公式应用于 $A^*$把 $A^*$ 看作一个新的矩阵 $B$,根据性质 $B B^* = |B| E$,我们可以写出:$$A^* (A^*)^* = |A^*| E$$代入 $|A^*|$ 的公式我们知道 $|A^*| = |A|^{n-1}$,将其代入上式:$$A^* (A^*)^* = |A|^{n-1} E$$两边左乘 $A$为了利用 $A A^* = |A| E$ 消除左边的 $A^*$,我们在等式两边同时左乘矩阵 $A$:$$A \left[ A^* (A^*)^* \right] = A \left( |A|^{n-1} E \right)$$结合律与化简利用矩阵乘法结合律 $(A A^*) (A^*)^* = |A|^{n-1} A$。因为 $A A^* = |A| E$,代入得:$$(|A| E) (A^*)^* = |A|^{n-1} A$$即:$$|A| (A^*)^* = |A|^{n-1} A$$得出结论当 $|A| \neq 0$ 时,我们可以从等式两边约去一个数 $|A|$:$$(A^*)^* = \frac{|A|^{n-1}}{|A|} A = |A|^{n-2} A$$

$Ax=0$ 显然能推出 $A^TAx=0$(两边左乘 $A^T$ 即可)
这个性质直接导致了两个非常重要的考研结论:同解性质:方程组 $Ax=0$ 和 $A^TAx=0$ 的解完全一样(同解)。秩相等性质(Rank):因为解空间一样大,根据“秩-零度定理”,矩阵的秩也必须一样。$$r(A) = r(A^TA) = r(A^T)$$

如果有两个矩阵,它们的特征值、行列式、迹都一样,但你怀疑它们不相似,就用$$r(A-kE) = r(B-kE)$$检查:看它们对应特征值的秩是否相等。如果不等,直接判死刑(不相似)。
\newpage
\printbibliography
\end{document}
